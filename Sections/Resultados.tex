\label{chapter:resultados}
\par
Neste capitulo será apresentado os resultados iniciais da ferramenta, bem como o benchmark e o método de análise utilizado.

\section{Método de análise utilizado.}
O teste apresentado neste capitulo, tem o objetivo de apresentar o potencial ferramenta de tradução, V2C, bem como, suas limitações e o impacto desta ferramenta no método. Baseado nestes objetivo foram desenvolvidas as seguintes questões de pesquisa:
\begin{enumerate}
    \item A ferramenta V2C é capaz de traduzir todos códigos do benchmark?
    \item Caso não traduza, é possivél alterar o código de modo que possa ser traduzido?
    \item Quais são os impactos de usar esta ferramenta no desenvolvimento do método?
\end{enumerate}
\par
Para o teste da ferramenta V2C foram utilizados dois benchmark, sendo um dele desenvolvido pelo próprio autor e outro benchmark de fonte externa, desta forma totalizando 27 códigos. Os benchmark do do autor foi desenvolvido utilizado os conceitos básicos de VHDL apresentando códigos mais simples e o benchmark externo apresenta códigos mais complexos, disponivel no link: \textbf{http://www.pld.ttu.ee/~maksim/benchmarks/iscas99/vhdl/}.

\par
O motivo de utilizar um benchmark externo é testar o real desempenho da ferramenta, testando na apenas com códigos simples, como porta lógicas ou multiplexadores, mas também códigos que apresentem outras estruturas do VHDL, além das utilizadas no benchmark do autor, em outras palavras, em códigos mais complexo, as chance são maior de a ferramenta não conseguir realizar a tradução do código.

\todo[inline]{Demonstrar de que modo deve ser realizado o teste}
\par
Para o teste foi utilizado a execução direta da ferramenta para que o teste envolvesse apenas a ferramenta V2C e o benchmark. Para cada código foi realizado uma única execução do programa e gerado o resultado, caso a tradução ocorresse corretamente, um arquivo em linguaguem C é gerado e em caso de falha é apontado erro e a linha ao qual apresentou o problema. 

\par
A ferramenta não apresenta em exato qual a causa do erro, apenas a linha onde o mesmo ocorreu, logo foi necessário analisar cada caso de erro e expecular de acordo com a linha do erro e as informações existentes na documentação da ferramenta.

\section{Apresentação do resultado das traduções realizadas pela ferramenta de tradução.}
\par
Após a execução do benchmark, os resultado da obtidos são apresentados na \autoref{tabela_resultado}, formada pelas colunas \textbf{Id} que é uma identificação para cada código; \textbf{Nome do código}; \textbf{Status da tradução} que indica se a tradução ocorreu corretamente ou não e \textbf{Identificação do erro} que determina qual o erro apresentado na execução falha da tradução.

\begin{table}[H]
\centering
\caption{Tabela de apresentação dos resultados do benchmark}
\label{tabela_resultado}
\begin{tabular}{|l|l|l|l|}
\hline
Id & Nome do código     & Status da tradução & Identificação do erro \\ \hline
1  & B01.vhd            & Erro               & A01                   \\ \hline
2  & B02.vhd            & Erro               &                       \\ \hline
3  & B03.vhd            & Erro               & A02                   \\ \hline
4  & B04.vhd            & Erro               & A03                   \\ \hline
5  & B05.vhd            & Erro               & A02                   \\ \hline
6  & B06.vhd            & Erro               & A02                   \\ \hline
7  & B07.vhd            & Erro               & A01                   \\ \hline
8  & B08.vhd            & Erro               & A02                   \\ \hline
9  & B09.vhd            & Erro               & A01                   \\ \hline
10 & B10.vhd            & Erro               & A02                   \\ \hline
11 & B11.vhd            & Erro               & A03                   \\ \hline
12 & B12.vhd            & Erro               & A02                   \\ \hline
13 & B13.vhd            & Erro               & A03                   \\ \hline
14 & B14.vhd            & Erro               & A03                   \\ \hline
15 & B14\_1.vhd         & Erro               & A03                   \\ \hline
16 & B15.vhd            & Erro               & A02                   \\ \hline
17 & B15\_1.vhd         & Erro               & A02                   \\ \hline
18 & AND\_ent,vhd       & Traduzido          & -                     \\ \hline
19 & OR\_end.vhd        & Traduzido          & -                     \\ \hline
20 & NAND\_ent.vhd      & Traduzido          & -                     \\ \hline
21 & NOR\_ent.vhd       & Traduzido          & -                     \\ \hline
22 & XOR\_ent.vhd       & Traduzido          & -                     \\ \hline
23 & Comb1.vhd          & Traduzido          & -                     \\ \hline
24 & Dff.vhd            & Traduzido          & -                     \\ \hline
25 & Flipflop\_D.vhd    & Traduzido          & -                     \\ \hline
26 & Mux\_4to1\_top.vhd & Traduzido          & -                     \\ \hline
27 & Seq1.vhd           & Traduzido          &                       \\ \hline
\end{tabular}
\end{table}

\par
Os códigos com Id entre 1 e 17 foram os códigos do benchmark externo e os códigos entre 18 e 27 representam os códigos do benchmark do autor. Os erros apresentado na \autoref{tabela_resultado} foram classificados de modo a facilitar o entendimento da tabela e nos erros apresentados nos testes, sendo definidos como:
\begin{itemize}
    \item \textbf{A01:} Corresponde aos erros relacionados a alguma paravra reservada, no qual a ferramenta não foi capaz de identificar.
    \item \textbf{A02:} Corresponde a erros relacionados a estruturas do VHDL não reconhecidas pela ferramenta.
    \item \textbf{A03:} Corresponde a erros relacionados a intervalos negativos ou de ordem decrescente.
\end{itemize}

\par
Conforme demonstrado na \autoref{tabela_resultado}, os códigos do benchmark do autor foram todos traduzidos com sucesso e isso se deve pelo fato dos códigos terem sidos desenvolvidos conforme as limitações da ferramenta, demonstrando uma boa execução da ferramenta na função a qual foi desenvolvida.

\par
Em contra partida, todos os códigos do benchmark externo apresentaram erros. A principal ocorrência de erros deve-se a estruturas do VHDL não suportadas, que neste caso foi a estrutura \textit{bit\_vector}. No caso das palavras reservadas a principal foi a palavra \textbf{\textit{Constant}} que apareceu em alguns dos códigos e não sendo reconhecido pela ferramenta ocasionando o erro. E também erros nos intervalos da declaração de inteiros, onde intervalos decrescentes não eram interpretados pela ferramenta, ocasionando o erro.

\todo[inline]{Possivel solução para o caso negativo de solução}
\todo[inline]{Apresentar um nova tabela com os códigos consertados }
\todo[inline]{Mostrar o que já esta automatizado}
\par
