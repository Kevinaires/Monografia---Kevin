\label{chapter:resultados}
\par
Neste capitulo será apresentado os resultados iniciais da ferramenta, bem como o benchmark e o método de análise utilizado.

\section{Método de análise utilizado.}
O teste apresentado neste capitulo, tem o objetivo de apresentar o potencial das ferramentas de tradução, vhdl2vl e v2c, bem como, suas limitações e o impacto destas ferramenta no método. Os teste também buscam avaliar a ação do método como um todo, analisando pontos positivos e negativos. Baseado nestes objetivos foram desenvolvidas as seguintes questões de pesquisa:
\begin{enumerate}
    \item As ferramentas de tradução são capazes de traduzir todos códigos do benchmark?
    \item Caso não traduza, é possivél alterar o código de modo que possa ser traduzido?
    \item Quais são os impactos de usar esta ferramenta no desenvolvimento do método?
    \item Quais os limites apresentados pela ferramenta de analise ESBMC?
    \item Quais os limites apresentado pelo método?
\end{enumerate}
\par
Para o teste das ferramentas de tradução e de analise, mas também do método foi utilizado um benchmark composto por 30 códigos. Estes códigos são oriundos das seguintes fontes:
\begin{itemize}
    \item \textbf{Fonte própria do autor:} 10 códigos foram desenvolvidos pelo autor para utilização em testes das ferramenta. Estes códigos correspondem a portas lógicas, multiplexadores, entre outros códigos. 
    \item \textbf{ICAS99:} 10 códigos foram extraidos do benchmark externo ISCAS99. Apresentam um grau de complexidade maior que os códigos do autor, e desta forma buscar os limites das ferramentas. O benchmark esta disponivel no link: \textbf{http://www.pld.ttu.ee/~maksim/benchmarks/iscas99/vhdl/}. 
    \item \textbf{Benchmark Vhd2vl:} 10 códigos foram extraídos do exemplos existente junto com a ferramenta Vhd2vl. Este códigos apresentam diversos graus de conos xidade, buscando analisar não somente as limitações desta ferramenta, mas se é possível que estes código sejam analisados ao final.
\end{itemize}
Logo, 30 códigos de diversas fontes foram utilizados no teste do método. O motivo da utilização de outras fontes de benchmark é testar o real desempenho da ferramenta, testando não apenas com códigos simples, como porta lógicas ou multiplexadores, mas também códigos que apresentem outras estruturas do VHDL, além das utilizadas no benchmark do autor, em outras palavras, em códigos mais complexo, as chance são maior de a ferramenta não conseguir realizar a tradução do código.

\par
Os testes foram realizados da seguinte forma, era passado o código juntamente com o arquivo externo para ferramenta, cada código possuia um arquivo diferente, devido a cada código possuir suas entrada e saídas diferentes e também condições diferentes apra cada código. Todos os códigos são passado para ferramenta e verificado se todas as etapas foram concluídas, caso contrário é analisado em qual etapa o erro foi gerado e a possível causa do erro. Na parte de analise, caso o tempo de analise por SMT ultrapasse os 5 min, a analise é realizada atráves de indução K.
%======================================
%Apresentação dos resultados dos testes
%======================================
\section{Apresentação dos resultados dos testes}
Após a execução do benchmark, os testes resultado da obtidos são apresentados na \autoref{tab:tabela_resultado}, formada pelas colunas \textbf{Id} que é uma identificação para cada código; \textbf{Nome do código}; \textbf{vhd2vl} que apresenta o status para tradução da ferramenta vhd2vl e também para instrumentação que ocorre nesta etapa; \textbf{v2c} para o status da ferramenta de tradução e a adição das assertivas que ocorre nesta etapa; \textbf{SMT} para analise pelo ESBMC pelo SMT; e \textbf{Indução K} para analise do ESBMC para indução k.

\par
Faz-se necessário relatar o significado de alguns elementos dentro da tabela, no caso, \textbf{inb()} e \textbf{ini()}. O elemento \textbf{Inb()} representa a função \_\_VERIFIER\_NONDET\_INT() e o elemento \textbf{inb()} representa a função \_\_VERIFIER\_NONDET\_BOOL().

\par
Entre os códigos utilizado no benchmark, os seguintes resultados foram recolhidos:
\begin{itemize}
    \item 15 códigos passaram por todas as etapas, sende estes compostos pelo 10 códigos do autor e 5 códigos do ICAS99.
    \item 5 códigos foram traduzidos para Verilog, instrumentados e traduzidos para C, entretanto, devido a erros na tradução gerada, estes não puderam ser instrumentados para a analise pelo ESBMC.
    \item 5 códigos foram traduzido para Verilog e instrumentados, contudo não foram traduzidos para C. Em grande parte devido a não reconhecimento de estruturas por parte da ferramenta V2C.
    \item 5 códigos não puderam ser traduzido para Verilog. Principalemte devido a estruturas ignoradas ou não aceitas pela ferramenta Vhd2vl, por exemplo \textit{integer range}.
\end{itemize}

\par
Com base nestes resultados apresentados é possível apresentar uma melhora no método de tradução, mesmo que uma ferramenta tenha em adicional tenha sido colocada para este processo. Porém, assim como observado no métodos anterior, as ferramentas não possuem a abrangência de toda a linguagem de origem, com isso erros de tradução ocorreram ou mesmo a não tradução.

\par
