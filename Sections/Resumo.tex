O objetivo desse trabalho é a verificação de circuitos lógicos descritos em VHDL, adotando técnicas de verificação formal e transformações de código, buscando identificar funcionamento incorreto de determinado circuito. A análise é realizada através da inserção manual ou automática de assertivas contendo propriedades sobre um dado circuito analisado em VHDL, onde é aplicada transformação e instrumentação de código VHDL para C para utilização da técnica de \textit{Bounded Model Checking} (BMC) para exploração de todos os estados alcançáveis do circuito e assim determinar a ocorrência ou não de violação de propriedade descrita na assertiva inserida ao código. Durante os teste iniciais, foi desenvolvido uma ferramenta para automatizar o método proposto, onde foi constatado que a ferramenta de tradução de código VHDL para C apresentava limitações em relação as estruturas em VHDL, de modo a evitar que circuitos mais complexos pudessem ser analisados e assim restringindo a capacidade do método proposto neste trabalho. Positivamente, a ferramenta de BMC adotada, neste caso o ESBMC (\textit{Extended SMT-Based Bounded Model Checker}) apresentou resultados positivos, ou seja, validando as assertivas/propriedades de forma correta. Como trabalhos futuros, é necessário a busca de nova ferramenta de tradução e também a inserção automática das assertivas, tornando o método mais automático e como menor carga de interferência do desenvolvedor.
\vspace{\onelineskip}
\noindent
\par
\textbf{Palavras-chaves}: Transformação de código, Bounded Model Cheching, VHDL, Hardware, Assertivas.