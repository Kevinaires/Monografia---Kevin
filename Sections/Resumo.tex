O objetivo desse trabalho é a verificação de circuitos lógicos descritos em VHDL, adotando técnicas de verificação formal e transformações de código, buscando identificar funcionamento incorreto de determinado circuito. A análise é realizada através da inserção de assertivas contendo propriedades sobre um dado circuito analisado em VHDL, onde é aplicada transformação e instrumentação de código VHDL para C para utilização da técnica de indução-k para exploração dos estados alcançáveis do circuito e assim determinar a ocorrência ou não de violação de propriedade descrita na assertiva inserida ao código. Para etapa de tradução foi desenvolvido dois métodos de tradução, buscando a melhor tradução, através de uma única ferramenta ou através de várias ferramentas. Para a análise foi utilizada a mesma ferramenta, chamada ESBMC (\textit{Extended SMT-Based Bounded Model Checker}) que utilizou a técnica de indução-k para análise de todos os códigos. A ferramenta utilizou as pré e pós condições inseridas em cada código para analisar se elas foram violadas ou não.
%Durante os teste iniciais, foi desenvolvido uma ferramenta para automatizar o método proposto, onde foi constatado que a ferramenta de tradução de código VHDL para C apresentava limitações em relação as estruturas em VHDL, de modo a evitar que circuitos mais complexos pudessem ser analisados e assim restringindo a capacidade do método proposto neste trabalho. Positivamente, a ferramenta de BMC adotada, neste caso o ESBMC (\textit{Extended SMT-Based Bounded Model Checker}) apresentou resultados positivos, ou seja, validando as assertivas/propriedades de forma correta. 
Como trabalhos futuros, a inserção automática das assertivas torna o método mais independente do desenvolvedor, por tem uma menor carga de interferência. Também refinar ainda mais a tradução, buscando abranger cada vez mais elementos da linguagem VHDL.
\vspace{\onelineskip}
\noindent
\par
\textbf{Palavras-chaves}: Transformação de código, Bounded Model Cheching, VHDL, Hardware, Assertivas.