\label{chapter:consideracoes}
O trabalho aborda o desenvolvimento de um método para análise de circuitos lógicos descrito em VHDL através da aplicação de técnica de transformação de código e a técnica \textit{Bounded Model Cheking}. Este método tem como objetivo de explorar os estados alcançáveis no circuito, e dessa forma identificar erros que possam resultar em mal funcionamento do sistema.

\par
Entre as dificuldade encontradas, a principal está ligada a etapa de tradução de código. Devido a ferramenta utilizada inicialmente conter diversas limitações e  a mesma diminui a eficácia da ferramenta, pois a quantidade de estruturas que podem ser utilizadas é limitada e com isso limitando também o processo de criação de circuitos lógicos.

\par
Para trabalhos futuros é necessário buscar uma nova ferramenta ou método de tradução que tenha uma gama maior de tradução, seja em estruturas ou palavras reservadas, deste forma o método desenvolvido torna-se mais eficiente e gerando uma menor intervenção do usuário no código VHDL. A melhoria na tradução de código também beneficia a geração das assertivas automaticamente, pois com uma gama maior de estruturas suportadas, mais assertivas podem ser geradas.

\par
Além disso a geração automática das assertivas é outro ponto abordado, visto que na etapa atual ainda não foi implementado. Este recurso visa da liberdade ao desenvolvedor promovendo de maneira automática os teste sobre o código, não necessitando que uma intervenção, tornando a inserção manual apenas em casos específicos de testes.

 